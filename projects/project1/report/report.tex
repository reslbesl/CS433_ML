\documentclass[10pt,conference,compsocconf]{IEEEtran}

\usepackage{cite}
\usepackage{hyperref}
\usepackage{graphicx}	% For figure environment


\begin{document}
\title{Project Report - Discovering the Higgs Particle}

\author{
  Names\\
  \textit{Department of Computer Science, EPFL, Switzerland}
}

\maketitle

\begin{abstract}
This short report summarises our attempt to 
\end{abstract}


\section{Introduction}

The discovery of the Higgs boson at the Large Hadron Collider (LHC) at CERN in 2012~\cite{Aad2012} caused a wide-range media hype around the world~\cite{Guardian}. The hunt for the "God particle", which plays a fundamental role in the standard model of particle physics, preceding its ground-breaking discovery lasted for 48 years from Peter Higg's first publication that predicted its existence~\cite{Higgs1964} to the official confirmation by CERN scientists in 2013~\cite{CERN}.\\
There are many reasons why the search for the Higgs boson was one of the largest in modern science's history and lasted for half a decade. One of the many challenges the team of scientists working at the LHC had to overcome is that one can never directly observe a Higgs boson. The Higgs boson, like most particle types, is unstable and, immediately after being produced, undergoes a process called particle decay. This means that instead of searching for the particle itself the CERN scientists conducted an endless number of experiments to identify the unique traces of its decay. However, as collisions of protons produce a whole host of unstable particles, finding the "decay signature" of the Higgs boson is a challenging classification task.\\
In this short report, we describe an (amateur's) attempt to reproduce this uniquely fascinating hunt for the particle "revealing nature's secrets"~\cite{CERN}. Given access to an open-source dataset containing the decay signatures of collision events, we attempted to build an accurate and robust prediction model to distinguish the Higgs boson's decay trace from other particles.

\section{Background}
\subsection{Task Description}

\subsection{Data Description}
The dataset provided contained 25,000 labelled, training examples generated by the ATLAS full detector simulator~\cite{}.

\subsection{Classification Methods}

\section{Results}
Some beautiful plots here

\section{Discussion}
Say all the things we could have done but didn't do.

\bibliography{references}
\bibliographystyle{IEEEtran}


\end{document}
